\documentclass[12pt]{article}
\usepackage[russian]{babel}
\usepackage{indentfirst}
\usepackage{mathtools}
\usepackage{enumitem}
\usepackage[left=2cm, right=2cm, top=2cm, bottom=2cm, bindingoffset=0cm]
{geometry}

\begin{document}

\textbf{Белоброцкий Денис 4 курс 5 группа}
\\
\begin{center}
	{\Large Лабораторная работа №4}
\end{center} 
\begin{center}
	{\large \textbf{Метод Монте-Карло}}
\end{center} 
\begin{center}
	Вариант 2
\end{center}

	\section*{Постановка задачи}
	\par Вычислить значение интеграла, использую метод Монте-Карло. Сравнить полученные результаты с точным значением или со значением полученным с помощью математических пакетов.
	\section*{Теория}
	\par \textbf{Метод Монте-Карло} 
	\par Для вычисления значения интеграла $ I = \int\limits_{\Omega} f(\overline{x}) d\overline{x} $ с помощью метода Монте-Карло применяется следующий алгоритм. Генерируется $ n $ случайных величин равномерно распределённых в области интегрирования $ \Omega $ и находится приближённое значение интеграла по следующей формуле:
	$$ I \approx Q_n \equiv \frac{V}{n} \sum\limits_{i=1}^n f(\overline{x}_i) $$
	\par Где $ V $ объём области интегрирования $ \Omega $.
	\par По закону больших чисел:
	$$ \lim\limits_{n\to \infty} Q_n = I $$

\end{document}